\documentclass[12pt,a4paper,landscape]{article}
\usepackage{luacode}
\usepackage{caption}
\usepackage[margin=0pt]{geometry}% 
\usepackage[table]{xcolor}
\usepackage{floatrow}
\floatsetup[table]{capposition=top}
\usepackage{tikz}
\usepackage{xcolor}

\pagecolor{black}

\begin{document}

\begin{luacode*}
	N = 100
		dt = 0.5
		    G = 1
			p = {
				{{14.5,-10.5},{0,0},2,10,"yellow"},
					{{7,-8.5},{0,-2},0.5,1,"blue"},
					    {{22, -12.5},{0,2},0.5,1,"red"},
						{{14.5,-14.5},{2,0},0.5,1,"green"},
						    {{14.4,-6.5},{-2,0},0.5,1,"pink"}
							}
							    nb = 5
								f = {{0,0},{0,0},{0,0}}
								    for k = 1,N do
											for e=1,nb do
														f[e] = {0,0}
																	for i=1,nb do
																					if e ~=i then
																										vx = p[e][1][1] - p[i][1][1]
																															vy = p[e][1][2] - p[i][1][2]
																																				d = math.sqrt(vx*vx + vy*vy)
																																									force = dt * G * p[i][4] / d
																																														f[e][1] = f[e][1] - ((force*vx)/d)
																																																			f[e][2] = f[e][2] - ((force*vy)/d)
																																																							end
																																																										end
																																																												end

																																																													ch = [[
																																																														\newpage
																																																																	\begin{tikzpicture}
																																																																		\draw (0,0) -- (30,-20);
																																																																					]]
																																																																						for e=1,nb do
																																																																							    p[e][2][1] = p[e][2][1] + f[e][1]
																																																																									p[e][2][2] = p[e][2][2] + f[e][2]
																																																																										    p[e][1][1] = p[e][1][1] + p[e][2][1]
																																																																												p[e][1][2] = p[e][1][2] + p[e][2][2]
																																																																													    ch = ch..[[
																																																																														    \fill[]]..p[e][5]..[[] (canvas cs:x=]]..p[e][1][1]..[[cm,y=]]..p[e][1][2]..[[cm) circle (]]..p[e][3]..[[);
																																																																																	    ]]
																																																																																		    end
																																																																																			    ch = ch..[[
																																																																	\end{tikzpicture}
																																																																																						    ]]
																																																																																							    tex.sprint(ch)
																																																																																								end
\end{luacode*}

\end{document}
